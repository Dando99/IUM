\documentclass{article}
\usepackage{graphicx}
\usepackage[utf8]{inputenc}
\usepackage[a4paper, total={6in, 8in}]{geometry}
\usepackage{fancyhdr}
\usepackage[hidelinks]{hyperref}
\usepackage{subcaption}

\pagestyle{fancy}
\fancyhf{}
\fancyfoot[L]{Alessandro Dori}
\fancyfoot[R]{\thepage}
\fancyhead[L]{Interazione Uomo Macchina}
\fancyhead[R]{SmartCycle}
\renewcommand{\headrulewidth}{0.4pt}
\renewcommand{\footrulewidth}{0.4pt}

\begin{document}
\begin{center}
    \Huge Progetto Interazione Uomo Macchina
        \vspace{0.5cm}

        \large A.A. 2024/2025 - Alessandro Dori 1843237
        \vspace{1cm}

        \large \textsf{\textbf{SmartCycle}}

        \includegraphics[width=0.3\textwidth, trim=110 110 110 110, clip]{img/SM.png}
\end{center}

\tableofcontents
\newpage

\section{Introduzione}
L'obiettivo di questo documento è raccogliere le statistiche ottenute dai questionari per il progetto SmartCycle.
Questo documento servirà come riferimento per le fasi successive del progetto, fornendo un'analisi delle risposte raccolte.

\section{Domande dei Questionari}
\begin{enumerate}
    \item Qual è la tua età?
    \item Qual è la tua occupazione attuale?
    \item Con chi vivi attualmente?
    \item Quanto tempo dedichi alla spesa alimentare durante la settimana?
    \item Quante volte cucini a settimana?
    \item Quante volte ti capita di buttare cibo a settimana?
    \item Quante volte ordini cibo a settimana?
    \item Quanto è importante per te ridurre gli sprechi alimentari?
    \item Quanto sei disposto a spendere per acquistare cibo invenduto?
    \item Quanto spesso utilizzi applicazioni per acquistare cibo invenduto?
\end{enumerate}

\newpage
\section{Analisi dei Questionari}

\begin{figure}[h]
    \centering
    \begin{subfigure}{0.40\textwidth}
        \centering
        \includegraphics[width=\textwidth]{img/eta.png}
        \caption{Distribuzione delle età dei partecipanti}
    \end{subfigure}
    \hfill
    \begin{subfigure}{0.40\textwidth}
        \centering
        \includegraphics[width=\textwidth]{img/occupazione.png}
        \caption{Distribuzione delle occupazioni dei partecipanti}
    \end{subfigure}
\end{figure}

\begin{figure}[h]
    \centering
    \begin{subfigure}{0.40\textwidth}
        \centering
        \includegraphics[width=\textwidth]{img/composizione_familiare.png}
        \caption{Distribuzione della composizione familiare}
    \end{subfigure}
    \hfill
    \begin{subfigure}{0.40\textwidth}
        \centering
        \includegraphics[width=\textwidth]{img/tempo_spesa.png}
        \caption{Tempo dedicato alla spesa alimentare}
    \end{subfigure}
\end{figure}

\begin{figure}[h]
    \centering
    \begin{subfigure}{0.40\textwidth}
        \centering
        \includegraphics[width=\textwidth]{img/frequenza_cucina.png}
        \caption{Frequenza di cucina settimanale}
    \end{subfigure}
    \hfill
    \begin{subfigure}{0.40\textwidth}
        \centering
        \includegraphics[width=\textwidth]{img/frequenza_buttare.png}
        \caption{Frequenza di cibo buttato settimanale}
    \end{subfigure}
\end{figure}

\begin{figure}[h]
    \centering
    \begin{subfigure}{0.40\textwidth}
        \centering
        \includegraphics[width=\textwidth]{img/frequenza_ordine.png}
        \caption{Frequenza di ordine cibo settimanale}
    \end{subfigure}
    \hfill
    \begin{subfigure}{0.40\textwidth}
        \centering
        \includegraphics[width=\textwidth]{img/graphic2.png}
        \caption{Importanza della riduzione degli sprechi alimentari}
    \end{subfigure}
\end{figure}

\begin{figure}[h]
    \centering
    \begin{subfigure}{0.40\textwidth}
        \centering
        \includegraphics[width=\textwidth]{img/graphic3.png}
        \caption{Importanza della riduzione degli sprechi alimentari}
    \end{subfigure}
    \hfill
    \begin{subfigure}{0.40\textwidth}
        \centering
        \includegraphics[width=\textwidth]{img/graphic4.png}
        \caption{Disponibilità a spendere per cibo invenduto}
    \end{subfigure}
\end{figure}

\begin{figure}[h]
    \centering
    \begin{subfigure}{0.40\textwidth}
        \centering
        \includegraphics[width=\textwidth]{img/graphic.png}
        \caption{Frequenza di utilizzo di applicazioni per cibo invenduto}
    \end{subfigure}
\end{figure}

\clearpage
\section{Conclusioni}
Dai risultati dei questionari emerge che la maggior parte delle persone intervistate non conosce applicazioni come Too Good To Go e Olio.
Una buona parte delle persone intervistate butta cibo almeno una volta a settimana.
Per la maggior parte delle persone è molto importante ridurre lo spreco alimentare.
Molte persone sono disposte a spendere una fascia di prezzo tra i 5 e i 10 euro per acquistare cibo invenduto.

\end{document}
