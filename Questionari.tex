\documentclass[a4paper,12pt]{article}
\usepackage[utf8]{inputenc}
\usepackage{graphicx}
\usepackage{geometry}
\usepackage{titling}
\usepackage[italian]{babel}
\usepackage[hidelinks]{hyperref}

\geometry{margin=1in}

\title{Progetto SmartCycle}
\author{Alessandro Dori}
\date{28 Novembre 2024}

\begin{document}

\begin{titlepage}
    \centering
    \vspace*{2cm}
    
    \Huge
    \textbf{Progetto SmartCycle}
    \includegraphics[width=0.4\textwidth]{immagini/logoSapienza.png}
    
    \vspace{1.5cm}
    
    \LARGE
    Domande dei Questionari e Statistiche Ottenute
    
    \Large
    \textbf{Alessandro Dori 1843237}
    
    \vspace{0.8cm}
    
    \Large
    \today
    
    \vspace{2cm}
    
    \includegraphics[width=0.4\textwidth]{immagini/logoSapienza.png}
    
\end{titlepage}

\tableofcontents
\newpage

\section{Introduzione}
L'obiettivo di questo documento è raccogliere le statistiche ottenute dai questionari per il progetto SmartCycle.
Questo documento servirà come riferimento per le fasi successive del progetto, fornendo un'analisi delle risposte raccolte.

\section{Domande dei Questionari}
\begin{enumerate}
    \item Qual è la tua età?
    \item Qual è la tua occupazione attuale?
    \item Con chi vivi attualmente?
    \item Quanto tempo dedichi alla spesa alimentare durante la settimana?
    \item Quante volte cucini a settimana?
    \item Quante volte ti capita di buttare cibo a settimana?
    \item Quante volte ordini cibo a settimana?
    \item Quanto è importante per te ridurre gli sprechi alimentari?
    \item Quanto sei disposto a spendere per acquistare cibo invenduto?
    \item Quanto spesso utilizzi applicazioni per acquistare cibo invenduto?
\end{enumerate}

\end{document}