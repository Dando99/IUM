\documentclass{article}
\usepackage{graphicx}
\usepackage[utf8]{inputenc}
\usepackage[a4paper, total={6in, 8in}]{geometry}

\author{Alessandro Dori}
\date{\today}

\begin{document}
\hbox{\Huge Progetto Interazione Uomo Macchina}
\begin{center}
    \Huge A.A. 2024/2025
\end{center}

\begin{center}
    \begin{tabular}{c c}
        \includegraphics[width=0.2\textwidth]{img/logoSapienza.png} & \huge SmartCycle \\
    \end{tabular}
\end{center}

\section*{Obiettivo}
Il progetto nasce dall'idea di sviluppare un sistema digitale che aiuti a ridurre gli sprechi alimentari, ispirato al modello di Too Good To Go. 
L’idea principale è creare un sistema che permetta agli utenti di trovare e acquistare prodotti alimentari invenduti a prezzi ridotti, mettendoli in contatto con negozi, ristoranti e supermercati della loro zona.

L’obiettivo da un lato è di aiutare i commercianti a vendere cibi che altrimenti verrebbero sprecati, dall’altro offrire agli utenti un modo semplice per risparmiare e fare scelte sostenibili. 
Il sistema fornirà funzionalità come la ricerca per posizione, notifiche personalizzate e la possibilità di gestire facilmente prenotazioni.

Questo sistema vuole risolvere un problema pratico, ma anche sensibilizzare le persone verso un consumo più consapevole e rispettoso dell’ambiente, contribuendo a ridurre l’impatto negativo dello spreco alimentare.
\newpage
\section*{Domande per l'intervista}
\begin{enumerate}
    \item Nome?
    \item Età?
    \item Con chi vivi?
    \item Città di residenza?
    \item Quanto tempo dedichi alla preparazione dei pasti durante la settimana?
    \item Quanto tempo dedichi alla spesa alimentare durante la settimana?
    \item Quali difficoltà incontri nella gestione degli sprechi alimentari?
    \item Utilizzi già qualche sistema o applicazione per ridurre gli sprechi alimentari? Se si, quale? (Se non si utilizza alcun sistema, saltare alla sezione Domande per utenti che non utilizzano applicazioni per acquistare cibo invenduto)
    \item Quando è stata l'ultima volta che hai utilizzato un'applicazione per acquistare cibo invenduto?
    \item Quanto spesso ti capita di acquistare cibo invenduto tramite applicazioni o altri sistemi?
    \item L'ultima volta che hai utilizzato un'applicazione per acquistare cibo invenduto, hai riscontrato qualche problema? Se si, quali?
    \item In base a cosa decidi di utilizzare un'applicazione per acquistare cibo invenduto rispetto ad andare direttamente di persona?
    \item Quali sono i fattori che ti spingono a scegliere una pizzeria/hamburgeria o un ristorante rispetto ad un altro quando utilizzi queste applicazioni?
\end{enumerate}

\paragraph{Domande per utenti che non utilizzano applicazioni per acquistare cibo invenduto}
\begin{enumerate}
    \item Se no, come mai non hai mai utilizzato un'applicazione per acquistare cibo invenduto?
    \item Cosa ti spingerebbe a utilizzare un'applicazione per acquistare cibo invenduto?
    \item Quali sarebbero le motivazioni per cui sceglieresti di utilizzare un applicazione per acquistare cibo invenduto (ipotizzando che la conosca) rispetto ad andare direttamete di persona?
    \item Cosa ti spingerebbe a scegliere una pizzeria/hamburgeria o un ristorante rispetto ad un altro ipotizzando che tu utilizzassi queste applicazioni?
\end{enumerate}
\newpage
\section*{Questionario}
\begin{enumerate}
    \item Qual è la tua età?
    \item Qual è la tua occupazione attuale?
    \item Con chi vivi attualmente?
    \item Quanto tempo dedichi alla spesa alimentare durante la settimana?
    \item Quante volte cucini a settimana?
    \item Quante volte ti capita di buttare cibo a settimana?
    \item Quante volte ordini cibo a settimana?
    \item Quanto è importante per te ridurre gli sprechi alimentari?
    \item Quanto sei disposto a spendere per acquistare cibo invenduto?
    \item Quanto spesso utilizzi applicazioni per acquistare cibo invenduto?
\end{enumerate}

\end{document}
