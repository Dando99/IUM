\documentclass{article}
\usepackage{graphicx}
\usepackage[utf8]{inputenc}
\usepackage[a4paper, total={6in, 8in}]{geometry}

\author{Alessandro Dori}
\date{\today}

\begin{document}
\hbox{\Huge Progetto Interazione Uomo Macchina}
\begin{center}
    \Huge A.A. 2024/2025
\end{center}

\begin{center}
    \begin{tabular}{c c}
        \includegraphics[width=0.2\textwidth]{img/logoSapienza.png} & \huge SmartCycle \\
    \end{tabular}
\end{center}

\section*{Obiettivo}
Il progetto nasce dall'idea di sviluppare un sistema digitale che aiuti a ridurre gli sprechi alimentari, ispirato al modello di Too Good To Go. 
L’idea principale è creare un sistema che permetta agli utenti di trovare e acquistare prodotti alimentari invenduti a prezzi ridotti, mettendoli in contatto con negozi, ristoranti e supermercati della loro zona.

L’obiettivo da un lato è di aiutare i commercianti a vendere cibi che altrimenti verrebbero sprecati, dall’altro offrire agli utenti un modo semplice per risparmiare e fare scelte sostenibili. 
Il sistema fornirà funzionalità come la ricerca per posizione, notifiche personalizzate e la possibilità di gestire facilmente prenotazioni.

Questo sistema vuole risolvere un problema pratico, ma anche sensibilizzare le persone verso un consumo più consapevole e rispettoso dell’ambiente, contribuendo a ridurre l’impatto negativo dello spreco alimentare.

\section*{Domande per l'intervista}
\begin{enumerate}
    \item Quali sono le principali difficoltà che incontri nella gestione degli sprechi alimentari?
    \item Utilizzi già qualche sistema o applicazione per ridurre gli sprechi alimentari? Se si, quale?
    \item L'ultima volta che hai utilizzato un'applicazione per acquistare cibo invenduto, hai riscontrato qualche problema? Se si, quali?
    \item In base a cosa decidi di utilizzare un'applicazione per acquistare cibo invenduto rispetto ad andare direttamente in negozio?
    \item Quanto è importante per te il risparmio economico rispetto alla sostenibilità ambientale? Perchè?
\end{enumerate}

\end{document}
